\section{Motivation \& Summary of Original Paper}

\begin{frame}
    \frametitle{Motivation}
    Web browsing is one of the most widely-used applications used in today's Internet ecosystem. A number of metrics have been developed and used as benchmarks to accurately reflect the performance of web applications for users. \textbf{Measurements can differ greatly depending on a number of factors:}
    \begin{itemize}
        \item Surveyed Web Pages
        \item User Devices
        \item Browsers
        \item Tools (e.g. Selenium)
        \item Metrics
    \end{itemize}
    $\boldsymbol{\rightarrow}$ This diversity as well as the lack of clearly established standards lead to difficulties when quantifying performance. The original paper \cite{10.1007/978-3-030-15986-3_19} examined the effects of these ambiguities with regards to browsers, tools, and metrics and provided guidelines for future papers.
\end{frame}

\begin{frame}
    \frametitle{Summary of Original Paper}
	The original paper \cite{10.1007/978-3-030-15986-3_19} was structured as follows:
    \begin{enumerate}
        \item Introduction
        \item Metric Definitions \& Tools
        \item Survey of 15 Web Studies
        \item Methodology
        \item Identification of Pitfalls \& Guidelines for Future Papers
    \end{enumerate}
\end{frame}